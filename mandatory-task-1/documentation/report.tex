\documentclass[a4paper, 12pt]{article}
\usepackage[utf8]{inputenc}
\DeclareUnicodeCharacter{00B2}{\ensuremath{{}^2}}
\usepackage{listings}
\usepackage{graphicx}
\title{Obligatorisk innlevering 1}
\author{Sivert M. Skarning}
\date{Mai 2019}
\begin{document}
\maketitle
\section{Oppgave 1}
\paragraph{Oppgavebeskrivelse}
Implementer konvolusjon på egen hånd. Implementasjonen skal være generell slik at den kan anvendes på alle bilder og filtere. Det er greit å begrense implementasjonen til å fungere på gråskalabilder, men det er ikke påkrevd. Implementasjonen skal støtte minst to ulike former for utvidelse/padding. Rapporten skal inneholde dokumentasjon for at implementasjonen fungerer.
\subsection{Dokumentasjon}
Vedlagt ligger et script kalt convolution.py. Koden utfører konvolusjon på bilde man legger ved som argument. Scriptet gir brukeren mulighet til å velge filterstørrelse of hvilke verdier filteret har. Brukeren har også mulighet til å velge mellom to forskjellige paddingmuligheter, zeropadding og reflectionpadding. Zeropadding er implementert med egenprogramert funksjon navngitt zero-pad.
Reflectionpad blir implmentert med pakken numpy og er navngitt np.pad. Begge funksjonene fungerer hvor zero-pad gir svart ramme rundt generert bild og reflectionpad gir utvidelse av bilde. Uten padding ville bruk av flere filtere etterhvert krympe bilde betydlig.
Selve implmentasjonen er inneffektivt implementert og i praksis ville man brukt en annen algoritme for å gjøre en konvolusjon. O-notasjonen for min implementsjon vil være: $$O(n² * m²)$$

\lstinputlisting[caption=Convolve function, language=Python, breaklines=true]{convolve.py}

\subsection{Resultat}
For å teste konvolveringsalgoritmen har jeg valgt å påføre tre typer filtere på lena.png.
\begin{itemize}
   \item Laplacian sharpening
   \item Mean blure
   \item Gaussian blure
\end{itemize}

\begin{figure}[h]
  \centering
  \includegraphics[width=0.5\textwidth]{images/gaussian-filter}
  \caption{Gaussian filter}
  \label{fig:gaussian-filter}
\end{figure}

\begin{figure}[h]
  \centering
  \includegraphics[width=0.5\textwidth]{images/gaussian-blurred-lena}
  \caption{Lena konvolvert med filteret på Figur \ref{fig:gaussian-filter}}
  \label{fig:gaussian-blurred-lena}
\end{figure}

\end{document}